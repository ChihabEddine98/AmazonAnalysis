%%%%    CHANGER mathsrc de euler en mathcal en gardant celui de LaTeX
%\let\mathcaltmp=\mathcal
%\let\mathscrtmp=\mathscr
%\usepackage{euler}
%\let\mathcal=\mathscr
%\let\mathscr=\mathscrtmp
%%%%
\usepackage[mathscr]{euscript}

\newcommand{\tsc}{\textsc}				%style capital
\newcommand{\mcl}{\mathcal}				%style semi-cursif
\newcommand{\scr}{\mathscr}				%style cursif
\newcommand{\M}{\mathfrak{M}}			%M matrices
\newcommand{\N}{\mathbb{N}}				%N naturels
\newcommand{\Z}{\mathbb{Z}}				%Z entiers relatifs
\newcommand{\Q}{\mathbb{Q}}				%Q fractionnels
\newcommand{\R}{\mathbb{R}}				%R réels
\newcommand{\Rn}{\R^n}			%R vécteurs
\newcommand{\Rnn}{\R^{n\times n}}			%R matrices carrées
\newcommand{\Rbar}{\overline{\R}}		%R barre, réels incluant infini
\newcommand{\C}{\mathbb{C}}				%C complexes
\newcommand{\K}{\mathbb{K}}				%K C ou R
\renewcommand{\P}{\mathscr{P}}			%P parties
\renewcommand{\L}{\mathscr{L}}			%L linéaire
\newcommand{\CC}{\mathscr{C}}			%C fonctions continues
\newcommand{\dd}{\text{d}}				%Dérivé (dt)
\newcommand{\Sc}{\scr{S}}				%S réels
\newcommand{\Cc}{\scr{C}}				%S réels
\newcommand{\D}{\scr{D}}			%P parties
\newcommand{\Y}{\scr{Y}}			%P parties
\newcommand{\X}{\scr{X}}			%P parties
\newcommand{\SPP}{\scr{S}_{++}^n}		%matrice sym def pos
\newcommand{\Eps}{\mathcal{E}}              %Eps
\newcommand{\norm}[1]{\left\lVert#1\right\rVert}
\newcommand{\cad}{\textbf{i.e :}}


\newcommand*\circled[1]{\tikz[baseline=(char.base)]{\node[shape=circle,draw,inner sep=1.5pt] (char) {#1};}}	%Entourer un caractère
\newcommand{\bam}{\begin{addmargin}[0.5cm]{0cm}}	%Ajouter l'indentation
\newcommand{\eam}{\end{addmargin}}		%Retirer l'indentation
\newcommand{\pmoins}{\scalebox{1}[1.0]{\( - \)}}	%Moins moins long
\newcommand{\ppmoins}{\scalebox{0.5}[1.0]{\( - \)}}	%Moins bien moins long
\newcommand{\fois}{\!\times\!}			%Multiplier, espaces raccourcis
\newcommand{\exist}{\exists\,}			%Il existe, espace droit augmenté
\newcommand{\findem}{\begin{flushright}\vspace{-0.35cm}$\square$\vspace{-0.2cm}\end{flushright}}	%Carré de fin de démonstration
\newcommand{\DE}{\textsc{d.e.}}			%Division euclidienne
\newcommand{\av}{\textsc{a.v.}{}}		%Au voisinage
\newcommand{\im}{\text{Im}}				%Image
\newcommand{\Ker}{\text{Ker}}			%Noyau
\newcommand{\sev}{\textsc{s-e.v.}}		%Sous-espace vectoriel
\newcommand{\Kev}{$\mathbb{K}$-\textsc{ev}}	%K-espace vectoriel
\newcommand{\rg}{\text{rg}}				%Rang
\newcommand{\Tr}{\text{Tr}}				%Trace
\newcommand{\Vect}{\text{Vect}}			%Vect
\newcommand{\diag}{\text{diag}}			%Diagonale (matrice)
\newcommand{\Card}{\text{Card}}			%Cardinal
\newcommand{\Id}{\text{Id}}				%Identité
\newcommand{\id}{\text{id}}				%Identité
\newcommand{\ul}{\underline}			%Souligner
\newcommand{\ol}{\overline}				%Sus-ligner
\newcommand{\tr}{\prescript{t}{}}		%Transposée
\newcommand{\com}{\footnotesize\ttfamily}	%Commentaire
\newcommand{\scol}{\setlength{\arraycolsep}{5pt}}	%Largeur colonnes large
\newcommand{\vect}{\overrightarrow}		%Sus-flecher
\newcommand{\ds}{\displaystyle}			%Forcer grand style
\newcommand{\scrotmp}{\mathchoice{{\scriptstyle\mathcal{O}}} {{\scriptstyle\mathcal{O}}} {{\scriptscriptstyle\mathcal{O}}} {\scalebox{.5}{$\scriptscriptstyle\mathscr{O}$}}}
\newcommand{\scro}{\scalebox{.8}[1.0]{$\scrotmp$}}%petit o manuscrit
\newcommand{\dlim}{\ds\lim}				%Limite avec indice en dessous
\newenvironment{psm}{\left(\begin{smallmatrix}}{\end{smallmatrix}\right)}	%Mini matrice

\makeatletter							%Flèche extensible
\newcommand{\xxrightarrow}[2][]{\ext@arrow 0359\xrightarrowfill@{#1}{#2}}
\newcommand{\xrightarrowfill@}{\arrowfill@\relbar\relbar{\mathrel{\smash{\rightarrow}\vphantom{\rightarrow}}}}
\makeatother

\makeatletter
\renewcommand{\o}[2][]{\scro_{#1}{\left(#2\right)}} %petit o de
\renewcommand{\O}[2][]{\mathcal{O}_{#1}{\left(#2\right)}} %grand o de
\newcommand{\simm}[1]{\,\underset{\raisebox{0.2ex}[0pt][0pt]{\scalebox{0.8}{$#1$}}}{\scalebox{1.1}[0.9]{$\sim$}}\,}			%équivalent à/simmilaire à
\newcommand{\restr}[2][]{{\left.\kern-\nulldelimiterspace #1\vphantom{\big|} \right|_{\scriptscriptstyle{#2}}}} 				%restreint à
\newcommand{\comp}[1]{{#1}^{\mathsf{c}}}		%Complémentaire
\makeatother

%%%%    REMISE DE mathcal (cf debut du fichier)
%\let\mathcal=\mathcaltmp

\definecolor{primaryColor}{rgb}{0.113, 0.364, 0.631}
\definecolor{primaryColor}{rgb}{0.6, 0.15, 0.18}
\definecolor{primaryColor}{rgb}{0.2, 0.45, 0.75}
\definecolor{secondaryColor}{rgb}{0.6, 0.6, 0,9}

\newcommand{\Exo}[2]{\vskip0.3cm\noindent \textcolor{primaryColor}{\textbf{Exercice #1 : #2 }}
\vskip0.2cm}

\usepackage{lipsum}
\usepackage{fancyhdr}
\usepackage{xcolor}

\usepackage{etoolbox}
\usepackage{amsmath,amsfonts,amssymb}
\usepackage[T1]{fontenc}

\definecolor{baseColor}{rgb}{0, 0.4, 1}
\definecolor{primaryColor}{HTML}{5c5554}
\definecolor{secondaryColor}{HTML}{101291}
\definecolor{successColor}{rgb}{0.02, 0.8, 0.25}
\definecolor{failColor}{rgb}{1, 0, 0}
\definecolor{mentionColor}{HTML}{dee3de}

\newcommand{\mention}[1]{\colorbox{mentionColor}{\textbf{#1}}}

\usepackage{tikz}
\usepackage{fontawesome}